\documentclass[12pt]{article}
\usepackage{amsmath, amsthm}
\usepackage{amssymb}
\usepackage[small]{titlesec}
\newtheorem{theorem}{Theorem}
\theoremstyle{definition}
\newtheorem{definition}{Definition}[section]
\title{Bayes' theorem}
\begin{document}
The probability of an event, based on prior knowledge of conditions that might be related to the event.
\begin{definition}(Conditional Probability)
	\[
		P(A|B) = \frac{P(A\cap B)}{P(B)}, where P(B) > 0.
	\]
\end{definition}
\begin{definition}(Bayes's Theorem)
	\[
		P(A|B) = \frac{P(A)P(B|A)}{P(B)}, where P(B) > 0.
	\]
\end{definition}
\begin{proof}
	\begin{align*}
		P(A|B) &= \frac{P(A\cap B)}{P(B)} \\
		P(B|A) &= \frac{P(A\cap B)}{P(A)} \\
		P(A|B)P(B) &= P(A\cap B) = P(B|A)P(A) \\
		P(A|B) &= \frac{P(B|A)P(A)}{P(B)}
	\end{align*}
\end{proof}

\end{document}
